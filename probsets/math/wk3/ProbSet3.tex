\documentclass[letterpaper,12pt]{article}
\usepackage{array}
\usepackage{threeparttable}
\usepackage{geometry}
\geometry{letterpaper,tmargin=1in,bmargin=1in,lmargin=1.25in,rmargin=1.25in}
\usepackage{fancyhdr,lastpage}
\pagestyle{fancy}
\lhead{}
\chead{}
\rhead{}
\lfoot{\footnotesize\textsl{OSM Lab, Summer 2017, Math,Inner Product Space \#3}}
\cfoot{}
\rfoot{\footnotesize\textsl{Page \thepage\ of \pageref{LastPage}}}
\renewcommand\headrulewidth{0pt}
\renewcommand\footrulewidth{0pt}
\usepackage[format=hang,font=normalsize,labelfont=bf]{caption}
\usepackage{amsmath}
\usepackage{amssymb}
\usepackage{amsthm}
\usepackage{natbib}
\usepackage{setspace}
\usepackage{float,color}
\usepackage[pdftex]{graphicx}
\usepackage{hyperref}
\hypersetup{colorlinks,linkcolor=red,urlcolor=blue,citecolor=red}
\theoremstyle{definition}
\newtheorem{theorem}{Theorem}
\newtheorem{acknowledgement}[theorem]{Acknowledgement}
\newtheorem{algorithm}[theorem]{Algorithm}
\newtheorem{axiom}[theorem]{Axiom}
\newtheorem{case}[theorem]{Case}
\newtheorem{claim}[theorem]{Claim}
\newtheorem{conclusion}[theorem]{Conclusion}
\newtheorem{condition}[theorem]{Condition}
\newtheorem{conjecture}[theorem]{Conjecture}
\newtheorem{corollary}[theorem]{Corollary}
\newtheorem{criterion}[theorem]{Criterion}
\newtheorem{definition}[theorem]{Definition}
\newtheorem{derivation}{Derivation} % Number derivations on their own
\newtheorem{example}[theorem]{Example}
\newtheorem{exercise}[theorem]{Exercise}
\newtheorem{lemma}[theorem]{Lemma}
\newtheorem{notation}[theorem]{Notation}
\newtheorem{problem}[theorem]{Problem}
\newtheorem{proposition}{Proposition} % Number propositions on their own
\newtheorem{remark}[theorem]{Remark}
\newtheorem{solution}[theorem]{Solution}
\newtheorem{summary}[theorem]{Summary}
%\numberwithin{equation}{section}
\bibliographystyle{aer}
\newcommand\ve{\varepsilon}
\newcommand\boldline{\arrayrulewidth{1pt}\hline}

\begin{document}

\begin{flushleft}
   \textbf{\large{Math, Inner Product Space\#3}} \\[5pt]
   OSM Lab instructor, John, Van den Berghe \\[5pt]
   OSM Lab student, CHEN Anhua\\[5pt]
   Due Wednesday, July 10 at 8:00am
\end{flushleft}

\vspace{5mm}


\begin{enumerate}
	\item (4.2)\\
	D should take the form of 
	$\begin{bmatrix}
	   0 & 1 & 0\\
	   0 & 0 & 2\\
	   0 & 0 & 0
	\end{bmatrix} $\\
	$p_{D}(z) = det(zI - D) = z^{3}$. If $p_{D}(z) = 0$, it indicates that eigenvalue of $D$ is $0$ with the algebric multiplicity of 3. Also $\mathcal{N}(0I - D) = \mathcal{N}(-D) = span\{[x, 0, 0]^{T}\}$. The geometric multiplicity is $1$, since $dim(\mathcal{N}(-D)) = 1$.\\
	
	\item(4.4)\\
	Using 4.3, we could show if $(tr(A)^{2} - 4det(A))$ is non-negative, then the matrix only got real eigenvalues, otherwise, it only gets imaginary eigenvalues.\\
	i. If the matrix is Hermitian, the $A =\begin{bmatrix}
	   a & b \\
	   b & d 
	\end{bmatrix}  $, then $(tr(A)^{2} - 4det(A)) = (a - d)^{2} + 4b^{2} \geq = 0$\\
	ii. If the matrix is skew-Hermitian, then $\begin{bmatrix}
	   0 & b \\
	   -b & 0 
	\end{bmatrix} $. Therefore, $(tr(A)^{2} - 4det(A)) = -4b^{2} < 0$ for $b \neq 0$\\
	
	
	\item(4.6)\\
	Let $A_{n \times n}$ be an upper triangular matrix. $det(\lambda I - A) = 0 \implies \prod_{i = 1}^{n}(\lambda_{i} - a_{ii}) = 0$. Therefore, the eigenvalues of matrix $A$ are its diagonal elements.\\
	
	\item(4.8)\\
	i. To prove $\{ sin(x), cos(x), sin(2x), cos(2x) \}$ is the basis for V, we need to prove they are linearly independent. This is equivalent to prove that for $\forall x \in \mathbb{R}, asin(x) + bcos(x) + csin(2x) + dcos(2x) = 0 only when a = b = c = d = 0 $. When $x=0, b + d = 0$. When $x = \pi, -b + d = 0$. Therefore, $b = d = 0$. Also, when $x = \frac{\pi}{2}, a-d = 0 \implies a = 0$. Then $\forall x \in \mathbb{R}, dcos(2x) = 0 \implies d = 0$. This completes the proof that S is a basis for V.\\
	ii. $(asin(x) +bcos(x) +csin(2x) +dcos(2x))^{'} = (-b)sin(x) + acos(x) +(-2d)sin(2x) + 2ccos(x)$. So the matrix representation of D is $\begin{bmatrix}
	   0 & -1 & 0 & 0 \\
	   1 & 0 & 0 & 0 \\
	   0 & 0 & 0 & -2 \\
	   0 & 0 & 2 & 0
	\end{bmatrix} $\\
	iii. \\

	\item(4.13)\\
	Let $B = P^{-1}AP$, where $B$ is a diagonal matrix. Since $P$ is a non-singular matrix, it indicates that A is diagonalizable. According to the Theorem 4.3.7, A is also semisimple. Then columns of  $P$ are just the eigenvectors of $A$ and $B$ is a diagonal marix with eigenvalues on its diagonal. $\lambda = 1$ and $0.4$. Therefore, we let $P = \begin{bmatrix}
	   2 &    1\\
	   1 & -1 \\
	\end{bmatrix}$\\

	\item(4.15)\\
	We will use Theorem 4.3.7 and Proposition 4.3.10 to prove this. According to Theorem 4.3.7, $A$ is semisimple, therefore diagonalizable. We could write $D = P^{-1}AP$ where $D$ is a diagonalizable matrix with eigenvalues of $A$ on its diagonal. And columns of $P$ form the eigenspace of $A$. According to Proposition 4.3.10, we know that for $i = 1,2, \hdots n, D^{i} = P^{-1}A^{i}P)$. Also, becasue $P$ is a linear operator, we have $P^{-1}f(A)P = a_{0}I + a_{1}P^{-1}AP + \hdots + a_{n}P^{-1}A^{n}P = a_{0}I + a_{1}D + \hdots + a_{n}D^{n}$. Since we know that $D^{i} =  \begin{bmatrix}
	   \lambda_{1}^{i} &     & \\
	    & \ddots & \\
	   &     &  \lambda_{n}^{i} \\
	\end{bmatrix}$, Then $P^{-1}f(A)P = \begin{bmatrix}
	   \sum_{i = 0}^{n} a_{i}\lambda_{1}^{i} &     & \\
	    & \ddots & \\
	   &     &  \sum_{i = 0}^{n} a_{i}\lambda_{n}^{i} \\
	\end{bmatrix} =  \begin{bmatrix}
	   f(\lambda_{1}) &     & \\
	    & \ddots & \\
	   &     &  f(\lambda_{n})\\
	\end{bmatrix}$ \\

	\item(4.16)\\
	i. We have in 4.13 that $D = P^{-1}AP, where D = \begin{bmatrix}
	   1 &    0\\
	   0 & 0.4\\
	\end{bmatrix} $ and $P =  \begin{bmatrix}
	   2 &    1\\
	   1 & -1 \\
	\end{bmatrix}$. According to proposition 4.3.10, $A^{n} = PD^{n}P^{-1}$. So $\lim_{n \to \infty}A^{n} = \lim_{n \to \infty}PD^{n}P^{-1} = P (\lim_{n \to \infty}D^{n}) P^{-1} = P \begin{bmatrix}
	   1 &  0\\
	   0 & 0 \\
	\end{bmatrix}P^{-1} =  \begin{bmatrix}
	   \frac{2}{3} &   \frac{2}{3}\\
	    \frac{1}{3} &  \frac{1}{3} \\
	\end{bmatrix}$\\
	ii. It doesn't matter.\\
	iii. Since the eigenvalues of $A$ are $1$ and $0.4$, according to Theorem 4.3.12, the eigenvalues are $3+5*1 + 1^3 = 9$ and $3 + 5*0.4 + 0.4^3 = 5.064$. \\

	\item(4.18)\\
	Since the characteristic polynomial of $A$ and $A^{T}$ is the same, therefore, $A$ and $A^{T}$ have the same eigenvalues. If $\lambda$ is a eigenvalue of $A$, it's Also an eigenvalue of $A^{T}$. Therefore, $\exists $non-zero $x$, such that $A^{T}X = \lambda x \implies x^{T}A =\lambda x^{T}$.\\

	\item(4.20)\\
	Since A is Hermitian, $B = U^{H}AU \implies B^{H} = ( U^{H}AU)^{H} = (U^{H}A^{H}U) = (U^{H}AU) = B$.\\
	
	\item(4.24)\\
	i. If $A$ is Hermitian, \\

	\item(4.31)\\
	i. $||A||_{2} = \sup_{x \neq 0} \frac{||Ax||_{2}}{||x||_{2}}$\\
	   $=  \sup_{x \neq 0} \frac{||U\Sigma V^{H}x||_{2}}{||x||_{2}}$\\
	   $= \sup_{x \neq 0} \frac{||\Sigma V^{H}x||_{2}}{||x||_{2}}$ (invariant under multiplication by orthonormal matrix)\\
	   $= \sup_{x \neq 0} \frac{||\Sigma y||_{2}}{||Vy||_{2}}$ (let $y  = V^{H}x$)\\
	   $= \sup_{x \neq 0} \frac{||\Sigma y||_{2}}{||y||_{2}}$  (invariant under multiplication by orthonormal matrix)\\
	   $= \sup_{x \neq 0} \frac{(\sum_{i = 1}^{n} |\sigma_{i}y_{i}|^{2})^{\frac{1}{2}}}{(\sum_{i = 1}^{n} |y_{i}|^{2})^{\frac{1}{2}}}$\\
	   We let y =  $[1, 0, \hdots 0]^{H}$, then $||A||_{2} = \sigma_{1}$\\
	ii. Since, based on singular value decomposition, $A^{-1} = (U\Sigma V^{H})^{-1} = V \Sigma^{-1} U^{H}$. The diagonal elements of $\Sigma^{-1}$ are $\frac{1}{\sigma_{1}}, \frac{1}{\sigma_{2}}, \hdots, \frac{1}{\sigma_{n}}$, the biggest of which will be $\frac{1}{\sigma_{n}}$. Therefore, according to the result of i., we know that $||A^{-1}||_{2} = \frac{1}{\sigma_{n}}$.\\
	iii. It's easy to prove that $||A||_{2}^{H} = ||A||_{2}^{T} = ||A||_{2}$ since the $\Sigma$ of their SVD is the same. $A^{H}A = V\Sigma^{H}U^{H}U\Sigma V^{H} = V \Sigma^{H}\Sigma V^{H}$. Since the diagonal elements of $ \Sigma^{H}\Sigma$ are $\sigma_{1}^{2}, \sigma_{2}^{2}, \hdots, \sigma_{n}^{2}$. By using the results of i., we could know that $||A^{H}A|| = \sigma_{1}^{2}$.\\
	iv. $||UAV||_{2}^{2} = ||UAVV^{H}A^{H}U^{H}||_{2} = ||UAA^{H}U^{H}||_{2} = ||AA^{H}U^{H}||_{2} = ||(AA^{H}U^{H})^{H}||_{2} = ||UAA^{H}||_{2} = ||AA^{H}||_{2} = ||A||_{2}^{2}$\\

	\item(4.25)\\
	
	


\end{enumerate}

\vspace{25mm}



\end{document}
