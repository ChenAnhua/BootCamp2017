\documentclass[letterpaper,12pt]{article}
\usepackage{array}
\usepackage{threeparttable}
\usepackage{geometry}
\geometry{letterpaper,tmargin=1in,bmargin=1in,lmargin=1.25in,rmargin=1.25in}
\usepackage{fancyhdr,lastpage}
\pagestyle{fancy}
\lhead{}
\chead{}
\rhead{}
\lfoot{\footnotesize\textsl{OSM Lab, Summer 2017, ECON, DSGE \#1}}
\cfoot{}
\rfoot{\footnotesize\textsl{Page \thepage\ of \pageref{LastPage}}}
\renewcommand\headrulewidth{0pt}
\renewcommand\footrulewidth{0pt}
\usepackage[format=hang,font=normalsize,labelfont=bf]{caption}
\usepackage{amsmath}
\usepackage{amssymb}
\usepackage{amsthm}
\usepackage{natbib}
\usepackage{setspace}
\usepackage{float,color}
\usepackage[pdftex]{graphicx}
\usepackage{hyperref}
\hypersetup{colorlinks,linkcolor=red,urlcolor=blue,citecolor=red}
\theoremstyle{definition}
\newtheorem{theorem}{Theorem}
\newtheorem{acknowledgement}[theorem]{Acknowledgement}
\newtheorem{algorithm}[theorem]{Algorithm}
\newtheorem{axiom}[theorem]{Axiom}
\newtheorem{case}[theorem]{Case}
\newtheorem{claim}[theorem]{Claim}
\newtheorem{conclusion}[theorem]{Conclusion}
\newtheorem{condition}[theorem]{Condition}
\newtheorem{conjecture}[theorem]{Conjecture}
\newtheorem{corollary}[theorem]{Corollary}
\newtheorem{criterion}[theorem]{Criterion}
\newtheorem{definition}[theorem]{Definition}
\newtheorem{derivation}{Derivation} % Number derivations on their own
\newtheorem{example}[theorem]{Example}
\newtheorem{exercise}[theorem]{Exercise}
\newtheorem{lemma}[theorem]{Lemma}
\newtheorem{notation}[theorem]{Notation}
\newtheorem{problem}[theorem]{Problem}
\newtheorem{proposition}{Proposition} % Number propositions on their own
\newtheorem{remark}[theorem]{Remark}
\newtheorem{solution}[theorem]{Solution}
\newtheorem{summary}[theorem]{Summary}
%\numberwithin{equation}{section}
\bibliographystyle{aer}
\newcommand\ve{\varepsilon}
\newcommand\boldline{\arrayrulewidth{1pt}\hline}

\begin{document}

\begin{flushleft}
   \textbf{\large{ECON, DSGE\#1}} \\[5pt]
   OSM Lab instructor, Kerk Phillips\\[5pt]
   OSM Lab student, CHEN Anhua\\[5pt]
   Due Wednesday, July 21 at 8:00am
\end{flushleft}

\vspace{5mm}


\begin{enumerate}
	\item(Exercise 1)\\
	Through substituting $K_{t + 1}$ and $K_{t + 2}$ by $Ae^{z_{t}}K_{t}^{\alpha}$ and  $Ae^{z_{t + 1}}K_{t + 1}^{\alpha}$, we can find that $A = \alpha \beta$\\

	
	\item(Exercise 2)\\
	\begin{equation} \label{eq1}
	\begin{split}
	\text{Budget cosntraint:} \quad c_{t} = (1 - \tau)[w_{t}l_{t} + (r_{t} - \delta)k_{t}] + k_{t} + T_{t} - k_{t + 1}\\
	\end{split}
	\end{equation}
	\begin{equation} \label{eq2}
	\begin{split}
	\text{Intertemporal Euler:} \quad \frac{1}{c_{t}} = \beta E_{t}\{  \frac{1}{c_{t + 1}}[(r_{t + 1} - \delta)(1 - \tau) + 1]\} \\
	\end{split}
	\end{equation}
	\begin{equation} \label{eq3}
	\begin{split}
	\text{Consumption-leisurel Euler:} \quad \frac{\alpha}{1 - l_{t}} = \frac{1}{c_{t}}w_{t} (1 - \tau) \\
	\end{split}
	\end{equation}
	\begin{equation} \label{eq4}
	\begin{split}
	\text{Capital FOC:} \quad r_{t} = \alpha e^{z_{t}}k_{t}^{\alpha - 1}l_{t}^{1 - \alpha} \\
	\end{split}
	\end{equation}
	\begin{equation} \label{eq5}
	\begin{split}
	\text{Labor FOC:} \quad w_{t} = (1 - \alpha) e^{z_{t}}k_{t}^{\alpha}l_{t}^{ - \alpha} \\
	\end{split}
	\end{equation}
	\begin{equation} \label{eq6}
	\begin{split}
	\text{Government budget:} \quad \tau[w_{t}l_{t} + (r_{t} - \delta)k_{t}] = T_{t}  \\
	\end{split}
	\end{equation}
	\begin{equation} \label{eq7}
	\begin{split}
	\text{Law of motions:} \quad z_{t} = (1 - \rho_{z})\bar{z} + \rho_{z}z_{t - 1} + \epsilon_{t}^{z} \\
	\end{split}
	\end{equation}

	\item(Exercise 3)\\
	
	\item(Exercise 4)\\

	\item(Exercise 5)\\
	\begin{equation} \label{eq1}
	\begin{split}
	\text{Budget cosntraint:} \quad c_{t} = (1 - \tau)[w_{t}l_{t} + (r_{t} - \delta)k_{t}] + k_{t} + T_{t} - k_{t + 1}\\
	\end{split}
	\end{equation}
	\begin{equation} \label{eq2}
	\begin{split}
	\text{Intertemporal Euler:} \quad c_{t}^{-\gamma} = \beta E_{t}\{  c_{t + 1}^{-\gamma}[(r_{t + 1} - \delta)(1 - \tau) + 1]\} \\
	\end{split}
	\end{equation}
	\begin{equation} \label{eq3}
	\begin{split}
	\text{Capital FOC:} \quad r_{t} = \alpha k_{t}^{\alpha - 1} (l_{t}e^{z_{t}})^{1 - \alpha} \\
	\end{split}
	\end{equation}
	\begin{equation} \label{eq4}
	\begin{split}
	\text{Labor FOC:} \quad w_{t} = (1 - \alpha)k_{t}^{\alpha} l_{t}^{- \alpha} e^{z_{t}(1 - \alpha)} \\
	\end{split}
	\end{equation}
	\begin{equation} \label{eq5}
	\begin{split}
	\text{Government budget:} \quad \tau[w_{t}l_{t} + (r_{t} - \delta)k_{t}] = T_{t}  \\
	\end{split}
	\end{equation}
	\begin{equation} \label{eq6}
	\begin{split}
	\text{Law of motions:} \quad z_{t} = (1 - \rho_{z})\bar{z} + \rho_{z}z_{t - 1} + \epsilon_{t}^{z} \\
	\end{split}
	\end{equation}
	If $l_{t} = 1$, then the steady state version of these equations will be: \\
	\begin{equation} \label{eq1}
	\begin{split}
	\text{Budget cosntraint:} \quad \bar{c} = (1 - \tau)[\bar{w} + (\bar{r} - \delta)\bar{k}] + \bar{T}\\
	\end{split}
	\end{equation}
	\begin{equation} \label{eq2}
	\begin{split}
	\text{Intertemporal Euler:} \quad \bar{c}^{-\gamma} = \beta E\{  \bar{c}^{-\gamma}[(\bar{r} - \delta)(1 - \tau) + 1]\} \\
	\end{split}
	\end{equation}
	\begin{equation} \label{eq3}
	\begin{split}
	\text{Capital FOC:} \quad \bar{r} = \alpha \bar{k}^{\alpha - 1} e^{\bar{z}(1 - \alpha)} \\
	\end{split}
	\end{equation}
	\begin{equation} \label{eq4}
	\begin{split}
	\text{Labor FOC:} \quad \bar{w} = (1 - \alpha)\bar{k}^{\alpha} e^{\bar{z}(1 - \alpha)} \\
	\end{split}
	\end{equation}
	\begin{equation} \label{eq5}
	\begin{split}
	\text{Government budget:} \quad \tau[\bar{w} + (\bar{r} - \delta)\bar{k}] = \bar{T}  \\
	\end{split}
	\end{equation}
	When being solved algebrically, $\bar{k} =\alpha^{\frac{1}{1 - \alpha}}(\frac{1}{1 - \tau}(\frac{1}{\beta} - 1) + \delta)^{\frac{1}{\alpha - 1}} e^{\bar{z}}$. Taking the value given in the question, we get $\bar{k} = 0.00919466769781$\\
	

\end{enumerate}

\vspace{25mm}



\end{document}
